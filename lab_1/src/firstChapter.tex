\begin{center}
	{\textbf{\normalsize Получение дизассемблированного кода обработчика прерывания int 8h}}
\end{center}

Для выполнения лабораторной работы на виртуальную машину была поставлена операционная система Windows XP (32 бит).

Для определения адреса вектора из таблицы векторов прерываний нужно вычислить смещение.
Так как номер прерывания --- 8h, а длина far-адреса составляет 4, нужно умножить номер вектора на 4 и
перевести полученное значение в шестнадцатеричную систему.

Получившееся значение -- \textbf{20h}.

Для получения содержимого по адресу 0000:0020h, то есть адреса обработчика прерывания, используется программа-отладчик
\textbf{AFDPRO}.
Перейдя к адресу 0000:0020h, можно увидеть значения четырёх байт: \textbf{46 07 0A 02}.

Так как у байтов обратный порядок следования (little endian), нужно поменять порядок местами.
Итоговый начальный адрес обработчика прерывания int 8h --- \textbf{020A:0746}.

Получение дизассемблированного кода производится чс помощью утилиты sourcer.
Для получения листинга кода нужно задать начальный и конечный адреса.
Конец обработчика прерывания можно найти, зная, что код обработчика заканчивается командой \textbf{iret}.
По адресу {020A:07B0} находится команда jmp \$-164h.
По смещению -164h находится несколько команд, в числе которых iret по адресу \textbf{020A:06AC}.
Поэтому листинг кода выполнялся в два этапа: сначала получения кода от смещения 0746h до смещения 07B0h,
а затем - от 064Ch до 06ACh.
